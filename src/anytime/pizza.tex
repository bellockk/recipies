
\begin{recipe}{Pizza}{serves 4}{1 hours}
  \ingredient[500]{grams}{warm water (90\textdegree{}F - 95\textdegree{}F}
  \ingredient[800]{grams}{flour}
  \ingredient[1]{tablespoon}{sugar}
  Mix together until all the flour is incorporated and then let sit for 30 minutes.
  \ingredient[2.5]{teaspoons}{salt}
  \ingredient[.5]{teaspoon}{yeast}
  Add salt and yeast and mix for 5 minutes.
  Mix dry ingredients together, set aside.  Place water in mixing bowl with hook attachment.  At slow speed, slowly add the dry ingredients mixture until all is used.  Add oil.  Knead for 4 to 5 minutes.  Divide and place in zip lock bags for cold rise (2 to 3 days).  When ready to cook, pull dough out of fridge 1 hour ahead of time to allow it to warm to room temperature.  Heat stone to 500+ degrees, the hotter the better.
  Cook on stone for 5--7 minutes at 500 degrees.
  \ingredient[1]{tablespoon}{olive oil}
\end{recipe}

Tips:
\begin{itemize}
  \item Do not add yeast to water, mix all dry ingredients together and slowly add it to the water.
  \item Weigh ingredients, especially flour, instead of using measuring cups.
  \item Less is better with yeast.
  \item Only use Instant Dry Yeast.  Do not use Active Dry Yeast.
  \item Cold rise dough.  This means, let your dough rise in a refridgerator for 24--72 hours.
  \item Mix the oil in the last step.
  \item Do not use a rolling pin to level out dough, you want to handle the dough as little as possible between rise and bake.
  \item Use corn meal/flour mix on peel to keep dough sliding on the peel.  This will help get the pizza off the peel and on the stone.
  \item Water temperature should be cool to room temperature.
  \item If you want to use the dough the next day, knead a little more (slow speed for 8 to 10 minutes) or if you have time to let the dough rest for 3 days knead for 4 to 5 minutes, low speed or head knead.
  \item Bakers percents: 62\% Hydration, 0.4\% yeast, 1.5\% salt, 1.5\% oil, 1\% sugar with a thickness factor of 0.08.
\end{itemize}
